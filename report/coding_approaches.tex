\chapter{Coding Approaches}

\section{Coding Approaches to Cost Function}

When talking about TSP we refer to the cost as the distance between nodes. These type of distances are defined by TSPLIB(REFERENCE) and can be computed or precomputed and given as input inside the .tsp files. In our case all the instances will be complete graphs and we will only be considering the instances where the cost is not given but has to be computed. Having said this, there is more than one way to handle such costs. We explored three ways:

\begin{enumerate}
	\item \textbf{Basic Approach}: Compute the distance using the distance function each time the value is needed.
	\item \textbf{Matrix Approach}: Compute only at the beginning a matrix of size \textit{number-of-nodes}$^2$ containing the costs of all possible pairing of nodes and than access the matrix to obtain edge costs.
	\item \textbf{AVX Approach}: Same as the basic approach but with the added bonus of using special vectorized instructions capable of computing multiple costs at once.
\end{enumerate}

\section{Technical Comparison}

Each one of the approaches described above has some advantages over the others as well as some disadvantages. In this part we will analyze the pros and cons of each of this methods. \\
Starting with the Basic Approach we can say that it is by itself the easiest one to implement since it simply requires to call the cost function each time there is the need to know the cost of the edge between two nodes. The advantages of this approach are its simplicity and the amount of memory required. The downside is that most of the times cost computations require the use of the use of the square root function. This is indeed a computationally expensive operation that can harm execution time of the various algorithm implemented in this project.

The Matrix Approach doesn't actually complicate things: after constructing the matrix, obtaining the cost of an edge merely requires accessing the matrix using the coordinates of the nodes connected by that edge. The matrix is structured such that each row contains the distances between one node and all the other nodes (including the node itself). The same applies to its columns.(XXX COULD INCLUDE PICTURE) The main advantage here is that it doesn't matter how computationally intensive the cost function might be since it is only computed once, at the beginning, therefore the time it needed to get the cost of an edge is the time needed for a memory access. However the matrix is very big (\textit{number-of-nodes}$^2$ elements) therefore it requires a lot of memory when the tsp instances get big (an istance with 16384 nodes requires 1GB of ram), therefore a lot of cache misses are expected when reading costs in different rows with particularly big instances.

Finally the AVX method is actually the most complicated one. This is because AVX2 instructions are a special kind of instructions used by mothern cpus introduced in 2011\cite{avxWikipedia}. These are \textit{Single-Instruction-Multiple-Data}(SIMD) instructions that allow to perform vectorized operations, in the AVX2 case the vectors have size up to 256 bit\cite{avxWikipedia}. It's a direct extension of the basic approach, with the difference being that eight costs are computed at once, which is, on paper, a very big speedup. This improvement comes at a price, that is an increase in coding complexity: slightly more sophisticated data structures are needed and some of the specialized instructions have complicated names.

\section{Performance Comparison}

When considering the difference performance wise there are a few consideration to take into account, one of these is the fact that most of the instances considered in this project have an expensive cost function which greatly affects the linear variable in time complexity analysis of the algorithms that we will describe later on. So theoretically speaking the fastest approach should be the one that uses the cost matrix since reading a value should be faster than recomputing it from scratch over and over, and in practice this often happens
