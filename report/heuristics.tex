\chapter{Heuristics}

\section{Nearest Neighbor}

Nearest Neighbor (NN) is a greedy algorithm that very simple and intuive yet relatively effective, used to find a feasible solution to the Traveling Salesman Problem.
In order to build the Hamiltonian circuit the algorithm indicates to move iteratively to the closest unvisited neighbor.\\
The procedure is:

\begin{enumerate}
    \item select a node as starting node,
    \item find the closes unvisited neighbor of the current node (greedy move),
    \item move to the selected node and mark it as visited,
    \item repeat from step 2 until all the nodes are marked as visited
\end{enumerate}

As for the initialization \textit{NN} returns a feasible solution no matter what is the starting node, but the cost of the solution will 
depend on it, therefore it is a good idea to set up a time limit and restarting the algorithm from different nodes, while keeping the
best solution found so far.
While this method provides a solution that is a good starting point from then implementing some metaheuristic to refine it, this solution
is usually far from the optimal one. This is due to the fact that by choosing always the closest node as the next one, it's creating a
lot of crossing between edges, reducing the efficiency of the solution.
\\

\begin{algorithm}[H]
	\TitleOfAlgo{\textbf{Nearest Neighbor}}
    \SetKwInOut{Input}{Input}
    \SetKwInOut{Output}{Output}
    \Input{A graph $G(V,E)$, a cost function $c_{ij}$, and a starting vertex $s$}
    \Output{A tour $T$ of the graph of cost $b$}
    \BlankLine
    $T \gets \{s\}$\\
    $v \gets s$\\
    $b \gets 0$\\
    \While{$|T| < |V|$}{
        identify node $u \in V/T$ such that $c_{vu} \leq c_{vj}$, $\forall$ $j \in V/(T \cup {u})$\\
        $T \gets T \cup \{u\}$\\
        $b \gets b + c_{vu}$\\
        $v \gets u$\\
    }
    $b \gets b + c_{vs}$\\
    \Return $tour$, $b$\\
\end{algorithm}

\subsection{Implementation}
We have some options on how we want to run NN. First off we can decide whether to use GRASP or not, but we can also configure 
how many threads we want to use, if we want to compute the distances between points every time needed or use a matrix to store 
them, or use AVX functions when possible to improve the performances. We can also choose to use 2-opt or 3-opt after finding a 
solution to improve its cost, or we can just leave it as is.

By using the cost matrix to store the weight of all the edges we can speed up the computation, but this will require a tradeoff 
in memory, O(n²) more specifically, where n is the number of nodes in the instance. Considering that we are storing nodes in a 
matrix of float variables, this is still good for many applications, e.g. an instance of 10000 nodes will require 0.37 GB of 
memory. This changes when we take into consideration larger instances, like the largest one in TSPlib, which includes 85900 
nodes and would require just short of 27.5 GB of memory.
This is why we implemented the computation of the edge weight on the fly both normally and with AVX instructions, which allows 
us to find the solution of more complex instances.

The function that handles the application of the heuristic is NearestNeighbor, to which we pass both the instance of the 
problem and the time limit. The application of GRASP and the creation of the required number of threads is handled here.
We launch each thread with the function loopNearestNeighbor, which loops the application of applyNearestNeighbor selecting 
a different starting node every time until all nodes have been used or the time limit is reached.
At all time we keep track of the best solution found so far, which is then the one returned by NearestNeighbor. When the 
solution is return 2-opt and 3-opt will be applied as desired.

This heuristic presents two main problems: the first is that, been a greedy algorithm, at each iteration we always visit 
the next closest point and this leads to the creation of many intersection between the edges of the solution, which are 
clearly inefficient (we deal with this problem using 2-opt or 3-opt). The second one is that is a deterministic algorithm, 
which means that if we run it from the same starting node multiple times the solution will never change, so the number 
of total solutions obtainable is limited to the number of nodes.


\section{Extra Mileage}

The \textit{Extra Mileage} heuristic is another approach used to find approximate solutions to the Traveling Salesman Problem (TSP).
While similar to the Nearest Neighbor Heuristic, it comes with a slight variation in the selection process. 
In the Extra Mileage Heuristic, the algorithm tries to minimize the additional mileage (or distance) traveled in each step, 
rather than always selecting the nearest unvisited city.\\
The method starts from an existing subtour, which can be created arbitrarily, a 3-node subtour is enough to start the method.
At each iteration, we take into consideration every unvisited node in every possible position of the current subtour, and compute the
increase in cost.\\
Let $c : V \times V \rightarrow \mathbb{R}$ be the function that, given two nodes, returns the cost of the edge between them.\\
Then $\forall$ covered node $i$ and uncovered node $h$: $\Delta(i,h) := $ cost of visiting $h$ right after $i$:\\

\begin{equation}
    \Delta(i,h) = c(i,h) + c(h, succ[i]) - c(i, succ[i])
\end{equation}\\

We will then procede to pick the most cost-efficient option, hence, the one that increases the cost the least: $min_{i,h}\{\Delta(i,h)\}$.
The node is inserted into the correct position in the subtour, and this process continues until all nodes are marked as visited, i.e. until the tour is completed.\\ 

\begin{algorithm}[H]
	\TitleOfAlgo{\textbf{Extra Mileage}}
    \SetKwInOut{Input}{Input}
    \SetKwInOut{Output}{Output}
    \Input{A graph $G(V,E)$, a cost function $c_{ij}$, and a starting subtour $S$}
    \Output{A complete tour $T$ of the graph of cost $b$}
    \BlankLine
    $T \gets S$\\
    $b \gets$ cost of $S$\\
    \While{{$|T| < |V|$}}{
        identify node $w \in V/T$ such that $m^w_{uv} = c_{uw} + c_{wv} - c_{uv}$ is minumum $\forall$ $u,v \in T$, $u \neq v$\\
        insert $w$ between $u$, $v$ int $T$\\
        $b \gets b + c_{uw} + c_{wv} - c_{uv}$\\
    }
    \Return $tour$, $b$\\
    
\end{algorithm}

\subsection{Implementation}
Extra mileage, like nearest neighbor, is a "solution builder", it generates a feasible solution starting from just the 
instance of the problem. Hence, we structured the implementation of this heuristic in a similar concept, maintaining the 
same configuration options, which are the application of GRASP, the number of threads to use, the computation of the edges' 
weights on the fly or store them in a matrix, and the application of 2-opt or 3-opt.

A similar structure is also implemented for the way we apply extra mileage to the instance. The main function that handles 
all the processes is ExtraMileage, which receives the instance and the time limit, sets GRASP if needed and creates the 
required number of threads.
The threads will run the function runExtraMileage until the time limit, launching in loop the function applyExtraMileage, 
which is the actual function that computes the heuristic.
